\documentclass{article}
\usepackage{graphicx}

\title{Informe M\'etodos de Programaci\'on}
\date{04/2017}
\author{Paola Olgu\'in}

\graphicspath{ {imagenes/}}

\begin{document}

  \pagenumbering{gobble}

  \renewcommand*\contentsname{\'Indice}

  \maketitle{}

  \newpage

  \pagenumbering{arabic}

  \tableofcontents{}

  \section{Introducci\'on}
  Aqui va a ir la introduccion. Incluir informacion sobre Java, diagramas de flujo,
  diagramas de clase, diagramas de sucesion, programacion orientada a objetos.
  \\
  \begin{center}
  \includegraphics[scale = 0.5, angle = 180]{triangulo1.png}
  \end{center}

  \newpage
  \section{Descripci\'on del problema}

  \par
  El problema a solucionar es el de, dado un triángulo de n filas,
  formado por triángulos más pequeños de color negro o blanco, hallar si hay
  triángulos formados únicamente de triángulos blancos dentro del triángulo
  mayor. Si se encuentra más de uno, reconocer el mayor, e imprimir por pantalla
  que el triángulo es de área a; donde a es la suma de todos los blancos que
  componen el triángulo hallado.\par
  Para menor confusión, se hablará de triángulo mayor para
  referirnos al triángulo formado por los menores de ambos colores, triángulo
  blanco para referirnos al mayor triángulo formado únicamente por blancos, y
  triángulos menores para los de menor tamaño.\par
  Para poder analizar bien el problema, se tomó en cuenta ciertas
  consideraciones que se mencionarán a continuación:








\end{document}
