\documentclass[spanish]{article}
\usepackage{graphicx}
\usepackage[utf8]{inputenc}
\usepackage[spanish]{babel}

\title{Informe Métodos de Programación}
\date{04/2017}
\author{Paola Olgu\'in}

\graphicspath{ {imagenes/}}

\begin{document}

  \pagenumbering{gobble}

  \maketitle{}

  \newpage

  \pagenumbering{arabic}

  \tableofcontents{}

  \section{Introducción}
  Aqui va a ir la introduccion. Incluir informacion sobre Java, diagramas de flujo,
  diagramas de clase, diagramas de sucesion, programacion orientada a objetos.

  \subsection{Java}
  Blablablablabla

  \subsection{Diagramas}
  \subsubsection{Flujo}
  \subsubsection{Clase}
  \subsubsection{Sucesi\'on}
  \subsection{Programaci\'on Orientada a Objetos}
  \newpage
  \section{Descripci\'on del problema}

  \par
  El problema a solucionar es el de, dado un triángulo de n filas,
  formado por triángulos más pequeños de color negro o blanco, hallar si hay
  triángulos formados únicamente de triángulos blancos dentro del triángulo
  mayor. Si se encuentra más de uno, reconocer el mayor, e imprimir por pantalla
  que el triángulo es de área a; donde a es la suma de todos los blancos que
  componen el triángulo hallado.\par
  Para menor confusión, se hablará de triángulo mayor para
  referirnos al triángulo formado por los menores de ambos colores, triángulo
  blanco para referirnos al mayor triángulo formado únicamente por blancos, y
  triángulos menores para los de menor tamaño.\par
  Para poder analizar bien el problema, se tomó en cuenta ciertas
  consideraciones que se mencionarán a continuación:
    \begin{enumerate}
    \item Las filas son de un número impar de triángulos menores,
    lo cual implica que la mayor área posible va a ser \(n^2\), con \(n\) siendo el
    número de filas del triángulo mayor, y la menor área posible será 0,
    en el caso que el triángulo mayor se componga únicamente de triángulos
    negros. Esto puesto que $\sum_{i=0}^{n} 2{i}+1 = {n}^2$
    \item Teniendo lo anterior en mente, si en la fila \(n-1\) (considerando la
    primera como 0), nos encontramos un triangulito negro, sabemos enseguida
    que la mayor área ya no puede ser \({n}^2\), sino que \({(n-1)}^2\). Esto, sin embargo,
    no nos dice demasiado, siendo que el triángulo blanco puede formarse hacia
    arriba (hacia la base) o hacia abajo (la punta), y puede formarse por
    partes de filas sin considerarlas enteras.
    \item Retomando la consideración, puesto que las filas que componen el
    triángulo mayor son todas impares, ningún triángulo blanco puede tener
    una fila con un número par de triángulos menores.


    \end{enumerate}

  \begin{center}
  \includegraphics[scale = 0.5, angle = 180]{triangulo1}
  \end{center}


\end{document}
