\documentclass{article}
\usepackage{graphicx}
\usepackage{subcaption}


\title{Informe M\'etodos de Programaci\'on}
\date{04/2017}
\author{Paola Olgu\'in}

\begin{document}

  \pagenumbering{gobble}

  \maketitle{}

  \newpage

  \pagenumbering{arabic}

  \tableofcontents{}

  \section{Introducci\'on}
  Aqui va a ir la introduccion. Incluir informacion sobre Java, diagramas de flujo,
  diagramas de clase, diagramas de sucesion, programacion orientada a objetos.

  \subsection{Java}
  Blablablablabla

  \subsection{Diagramas}
  \subsubsection{Flujo}
  \subsubsection{Clase}
  \subsubsection{Sucesi\'on}
  \subsection{Programaci\'on Orientada a Objetos}
  \newpage
  \section{Descripci\'on del problema}

  \par
  El problema a solucionar es el de, dado un tri\'angulo de n filas,
  formado por tri\'angulos m\'as pequeños de color negro y blanco, hallar si hay
  tri\'angulos formados \'unicamente de tri\'angulos blancos dentro del tri\'angulo
  mayor. Si se encuentra m\'as de uno, reconocer el mayor, e imprimir por pantalla
  que el tri\'angulo es de \'area a; donde a es la suma de todos los blancos que
  componen el tri\'angulo hallado.\par

    \begin{center}
      \includegraphics[scale = 0.5, angle = 180]{triangulo1}
    \end{center}

  Para menor confusi\'on, se hablar\'a de tri\'angulo mayor para
  referirnos al tri\'angulo formado por los menores de ambos colores, tri\'angulo
  blanco para referirnos al mayor tri\'angulo formado \'unicamente por blancos, y
  tri\'angulos menores para los de menor tamaño.\par

  Para poder analizar bien el problema, se tom\'o en cuenta ciertas
  consideraciones que se mencionar\'an a continuaci\'on:
    \begin{enumerate}
    \item Las filas son de un n\'umero impar de tri\'angulos menores,
    lo cual implica que la mayor \'area posible va a ser \(n^2\), con \(n\) siendo el
    n\'umero de filas del tri\'angulo mayor, y la menor \'area posible ser\'a 0,
    en el caso que el tri\'angulo mayor se componga \'unicamente de tri\'angulos
    negros. Esto puesto que $\sum_{i=0}^{n} 2{i}+1 = {n}^2$
    \item Teniendo lo anterior en mente, si en la fila \(n-1\) (considerando la
    primera como 0), nos encontramos un triangulito negro, sabemos enseguida
    que la mayor \'area ya no puede ser \({n}^2\), sino que \({(n-1)}^2\). Esto, sin embargo,
    no nos dice demasiado, siendo que el tri\'angulo blanco puede formarse hacia
    arriba (hacia la base) o hacia abajo (la punta), y puede formarse por
    partes de filas sin considerarlas enteras.
    \item Retomando la consideraci\'on, puesto que las filas que componen el
    tri\'angulo mayor son todas impares, ning\'un tri\'angulo blanco puede tener
    una fila con un n\'umero par de tri\'angulos menores.
    \item Cada fila empieza y termina con un tri\'angulo hacia la punta, lo que
    implica que los \'indices pares de las filas apuntan en esa direcci\'on y
    los \'indices impares apuntan hacia la base. La direcci\'on de un tri\'angulo
    se define, entonces, por hacia donde apuntan los extremos de la base, o,
    en su defecto, hacia donde mira la punta.
    \item Cada fila m\'as grande, tiene dos elementos m\'as que la
    anterior.
    \item Se consideran como tri\'angulos \'unicamente los que ser\'ian un
    tri\'angulo que nosotros podamos reconocer visualmente. Por
    lo tanto en un c\'odigo como el de la () reconocer\'ia un
    \'area de uno, puesto que su representaci\'on gr\'afica no nos
    muestra un tri\'angulo. En cambio, uno como el de la ()
    reconocer\'ia un \'area de cuatro.
    \end{enumerate}

  \section{Algoritmos}
  \subsection{Iteraci\'on}


\end{document}
